\iffalse
########## Dokumentation ##########

++++ Allgemein ++++

- DATUM: 20.03.25 (letzte Änderung)
- DHGE Vorgaben             (https://www.dhge.de/DHGE/dam/jcr:6e5d8217-0965-42fe-8982-c7515c12f0b9/Hinweise_und_Empfehlungen_schr_Arb_Technik_Gera_2023.pdf)
- VSCode Integration        (https://marketplace.visualstudio.com/items?itemName=James-Yu.latex-workshop)
- Rechtschreibung (VSCode)  (https://marketplace.visualstudio.com/items?itemName=ltex-plus.vscode-ltex-plus)
- Gruvbox (W Rizz)          (https://marketplace.visualstudio.com/items?itemName=jdinhlife.gruvbox)

++++ Installation ++++
  
MacOS   (https://www.tug.org/mactex/)
Linux   (https://wiki.debian.org/Latex)
Windows (https://miktex.org/)

++++ Latex ++++

Latex   (https://www.latex-project.org/help/documentation/)

##################################
\fi

% ########## Dokument Einstellungen ########## %
\documentclass[a4paper, 11pt]{article}

% ##########  Pakete ########## %

% ++++++++ Allgemeines +++++++++ %

\usepackage[ngerman]{babel}

% ++++++++ Text +++++++++ %

\usepackage[utf8]{inputenc}
\usepackage[T1]{fontenc}
\usepackage[hidelinks]{hyperref}
\usepackage[nohyperlinks]{acronym}
\usepackage[ style=custom-alphabetic-verb,dateabbrev=false, minalphanames=3, minbibnames = 3, maxbibnames=3]{biblatex} %minalphanames: Erste Buchstaben der eersten drei namen in eckige Klammer, minbibnames: erste drei Namen in bibligraphy, maxbibnames: höchstens drei Namen in Bibliography
\usepackage{setspace, titlesec, tocloft, csquotes, fancyhdr, enumitem, pifont, footnote, arial, caption, amsmath, amssymb, amsthm, amsfonts, floatbytocbasic}
\usepackage{float, subcaption, mathptmx, ragged2e, tocbasic, scrwfile}

% ########## Zitiereinstellung #########%

\addbibresource{bib.bib}

%für internetdokumente muss in Zitavi "online" eingestellt werden

%entfernt Punkt hinter fußnote%
\makeatletter
\renewcommand{\bibfootnotewrapper}[1]{%
  \bibsentence#1}
\makeatother


% ++++++++  Grafiken +++++++++ %

\usepackage{wrapfig, graphicx, listings, geometry, tabularx, pdfpages, tikz, dirtree, pdflscape,onimage}
\usetikzlibrary{shapes.geometric, arrows}

% ++++++++  Schriftart +++++++++ %

\usepackage{arial}

% ########## Flowchart ########## %

\tikzstyle{startstop} = [rectangle, rounded corners, 
minimum width=3cm, 
minimum height=1cm,
text centered, 
draw=black, 
fill=red!30]

\tikzstyle{io} = [trapezium, 
trapezium stretches=true,
trapezium left angle=70, 
trapezium right angle=110, 
minimum width=3cm, 
minimum height=1cm, text centered, 
draw=black, fill=blue!30]

\tikzstyle{process} = [rectangle, 
minimum width=3cm, 
minimum height=1cm, 
text centered, 
draw=black, 
fill=orange!30]

\tikzstyle{decision} = [diamond, 
minimum height=1cm, 
text centered, 
text width=2.5cm,
draw=black, 
fill=green!30]
\tikzstyle{arrow} = [thick,->,>=stealth]

% ########## Seiten Einstellungen ########## %
\geometry{a4paper, left=3cm, right=2cm, top=2.5cm, bottom=2.5cm}
\setlength{\parindent}{0pt}
\setlength{\parskip}{6pt}
\onehalfspacing

% ########## Bilder ########## %

% Beschriftung formatieren
\captionsetup{justification=raggedright,singlelinecheck=false}

% Rahmen Dicke
\setlength{\fboxrule}{1pt}
\setlength{\fboxsep}{0pt}

% Bilder auf neuen seiten sind oben

\makeatletter
\setlength{\@fptop}{0pt}
\setlength{\@fpsep}{20pt}
\makeatother

% ########## Text ########## %
% Schriftart Arial
\renewcommand{\sfdefault}{ua1}
\renewcommand{\familydefault}{\sfdefault}

% Überschriften nicht fett
\titleformat{\section}
  {\normalfont\Large}
  {\thesection}{16pt}{}
\titleformat{\subsection}
  {\normalfont\large}
  {\thesubsection}{14pt}{}
\titleformat{\subsubsection}
  {\normalfont\large}
  {\thesubsubsection}{14pt}{}

% ########## Code Snippets ########## %

% Farben Syntax Highlighting
\definecolor{codeblue}{rgb}{0.0,0.3,0.8}
\definecolor{codegreen}{rgb}{0.0,0.5,0.0}
\definecolor{codegray}{rgb}{0.5,0.5,0.5}

\lstdefinestyle{code}{
  columns=fullflexible,
  numbers=left,
  xleftmargin=30px,
  frame=single,
  framerule=1pt,
  framexleftmargin=25px,
  commentstyle=\color{codegreen},
  keywordstyle=\color{codeblue},
  numberstyle=\tiny\color{codegray},
  basicstyle=\ttfamily\footnotesize,
  captionpos=b,
  keepspaces=true,
  showstringspaces=false,
}
\lstset{style=code}

% ########## Kopf-/ & Fußzeile ########## %

% Textstyle
\pagestyle{fancy}
\renewcommand{\headrulewidth}{0pt}
\fancyhf{}
\fancyfoot[R]{\thepage}
\addtolength{\skip\footins}{3pc}

% Textstyle Inhaltsverzeichnis
\fancypagestyle{sxoli1}{
  \fancyhf{}
  \pagenumbering{Roman}
  \rfoot{\thepage}
  \renewcommand{\headrulewidth}{0pt}
  \renewcommand{\footrulewidth}{0pt}
	\fancypagestyle{plain}{
    \rfoot{\thepage}
	}
}

% ########## Verzeichnis Einstellungen ########## %

\renewcommand{\cftpartleader}{\cftdotfill{\cftdotsep}}
\renewcommand{\cftsecleader}{\cftdotfill{\cftdotsep}}
\renewcommand{\cftfigaftersnum}{:}
\renewcommand{\cfttabaftersnum}{:}
\renewcommand{\cftsecfont}{}
\renewcommand{\cftsecpagefont}{}
\setlength{\cftsecnumwidth}{50px}
\setlength{\cftsecindent}{0px}
\setlength{\cftsubsecnumwidth}{50px}
\setlength{\cftsubsecindent}{0px}
\setlength{\cftsubsubsecnumwidth}{50px}
\setlength{\cftsubsubsecindent}{0px}

% ########## Anlagenverzeichnis Setup ########## %

\newcommand{\anlagenname}{anlage}
\newfloat{anlage}{!ht}{an}
\newcommand{\listofanlagenname}{Anlagenverzeichnis}
\newcommand{\listofanlagen}{%
  \listof{anlage}{\listofanlagenname}%
}
\floatname{anlage}{Anlage}

% ########## Dokument Beginn ########## %
\begin{document}

% Keine Seitenzahlen auf Deckblatt und Sperrvermerk
\pagenumbering{gobble}

% ########## Titel ########## %
% Hier Titel engeben:
\title{TITEL}
\makeatletter
\let\inserttitle\@title
\makeatother

% ########## Deckblatt ########## %
\begin{center}
  \textbf{\inserttitle}
  \vspace*{1cm}
\end{center}
\begin{center}
  \begin{tabular}{ r p{10cm} }
    Projektarbeit & {\LARGE\bf\hspace{0.15cm}I\hspace{1.225cm}II\hspace{1.05cm}III\hspace{1cm}IV}\\
    Nr. & {\LARGE\bf $\boxtimes$\hspace{1cm}$\square$\hspace{1cm}$\square$\hspace{1cm}$\square$}\\[1.5cm]
    & 02.05.25 \\[-0.5cm]
    vorgelegt am: & \hrulefill \\[1cm]
    & Musterman, Max \\[-0.5cm]
    von: & \hrulefill \\[1cm]
    & [Nr] \\[-0.5cm]
    Matrikelnr: & \hrulefill \\[1cm]
    & Gera \\[-0.5cm]
    DHGE Campus: & \hrulefill \\[1cm]
    & [Studienbereich] \\[-0.5cm]
    Studienbereich & \hrulefill \\[1cm]
    & [Studiengang] \\[-0.5cm]
    Studiengang: & \hrulefill \\[1cm]
    & [Kurs] \\[-0.5cm]
    Kurs: & \hrulefill \\[1cm]
    & [Ausbildungsstätte] \\[-0.5cm]
    Ausbildungsstätte: & \hrulefill \\[1cm]
    & Musterman, Max \\[-0.5cm]
    Betreuer: & \hrulefill \\[1cm]
    & Musterman, Max \\[-0.5cm]
    & \hrulefill \\[1cm]
 \end{tabular}
\end{center}
\newpage

% ########## Sperrvermerk ########## %
\begin{center}
  \vspace*{5.5cm}
  {\LARGE\bf Sperrvermerk}
  \vspace*{1cm}
\end{center}

Die vorgelegte Projektarbeit mit dem Titel \inserttitle basiert auf internen, vertraulichen Daten und Informationen des Unternehmens BETRIEB.

Diese Projektarbeit darf nur vom Erst- und Zweitgutachter sowie berechtigten Mitgliedern des Prüfungsausschusses eingesehen werden.
Eine Vervielfältigung und Veröffentlichung der Projektarbeit ist auch auszugsweise nicht erlaubt.

Die Vervielfältigung und Veröffentlichung der Projektarbeit sowie die Einsichtnahme durch Dritte bedarf der ausdrücklichen
Zustimmung des Verfassers und des Unternehmens.
\newpage

% Römische Seitenzahl
\pagenumbering{Roman}


% ########## Inhaltsverzeichnis ########## %
\phantomsection
\section*{Inhaltsverzeichnis}
\pagestyle{sxoli1}
\renewcommand\contentsname{}
\tableofcontents
Ehrenwörtliche Erklärung
\newpage

% ########## Abbildungsverzeichnis ########## %
\phantomsection
\section*{Abbildungsverzeichnis}
\renewcommand\listfigurename{}
\setlength{\cftfigindent}{0em}
\setlength{\cftfignumwidth}{6.5em}
\renewcommand{\cftfigpresnum}{Abbildung }
\addcontentsline{toc}{section}{Abbildungsverzeichnis}
\listoffigures
\newpage

% ########## Tabellenverzeichnis ########## %
\phantomsection
\section*{Tabellenverzeichnis}
\renewcommand\listtablename{}
\setlength{\cfttabindent}{0em}
\setlength{\cfttabnumwidth}{5.5em}
\renewcommand{\cfttabpresnum}{Tabelle }
\addcontentsline{toc}{section}{Tabellenverzeichnis}
\listoftables
\newpage

% ########## Abkürzungsverzeichnis ########## %
\phantomsection
\section*{Abkürzungsverzeichnis}
\addcontentsline{toc}{section}{Abkürzungsverzeichnis}
\begin{acronym}[**********]
  \acro{DHGE}{Duale Hochschule Gera Eisenach}
  \acro{SQL}{Structured Query Language}
  \acro{Bash}{Bourne-again shell}
  \acro{JDK}{Java Development Kit}
  \acro{VM}{Virtuelle Maschine}
 \end{acronym}
\newpage

% Falls Plural abweichend
\acrodefplural{VM}[VMs]{Virtuelle Maschinen}

% Letzte Seite mit römischer Seitenzahl speichern & Arabische Zahlen verwenden
\newcounter{savepage}
\setcounter{savepage}{\number\value{page}}
\pagenumbering{arabic}

% ########## Text ########## %

\section{Einleitung}

\subsection{Fließtext}
Lorem ipsum dolor sit amet, consetetur sadipscing elitr, sed diam nonumy eirmod tempor invidunt ut labore et dolore magna aliquyam erat, sed diam voluptua. At vero eos et accusam et justo duo dolores et ea rebum. Stet clita kasd gubergren, no sea takimata sanctus est Lorem ipsum dolor sit amet. Lorem ipsum dolor sit amet, consetetur sadipscing elitr, sed diam nonumy eirmod tempor invidunt ut labore et dolore magna aliquyam erat, sed diam voluptua. At vero eos et accusam et justo duo dolores et ea rebum. Stet clita kasd gubergren, no sea takimata sanctus est Lorem ipsum dolor sit amet. Lorem ipsum dolor sit amet, consetetur sadipscing elitr, sed diam nonumy eirmod tempor invidunt ut labore et dolore magna aliquyam erat, sed diam voluptua. At vero eos et accusam et justo duo dolores et ea rebum. Stet clita kasd gubergren, no sea takimata sanctus est Lorem ipsum dolor sit amet.
Duis autem vel eum iriure dolor in hendrerit in vulputate velit esse molestie consequat, vel illum dolore eu feugiat nulla facilisis at vero eros et accumsan et iusto odio dignissim qui blandit praesent luptatum zzril delenit augue duis dolore te feugait nulla facilisi. Lorem ipsum dolor sit amet, consectetuer adipiscing elit, sed diam nonummy nibh euismod tincidunt ut laoreet dolore magna aliquam erat volutpat.

\textit{kursiv text}    \textbf{fetter text}

\newpage

\subsection{Zitate}
Ein direktes Zitat "`ist ein Zitat das sehr direkt ist."'\footnotemark
\footnotetext{\autocite{DIN.2008}}

dolore magna aliquyam erat, sed diam voluptua.\footnotemark
\footnotetext{Vgl. \autocite{Fowler.2006} \autocite{GitLab.2025}} 

est Lorem ipsum dolor sit amet.\footnotemark
\footnotetext{Ebd.}

\subsection{Formeln und mathematische Darstellung}

\subsubsection{Mathematische Grundsymbole}

\href{https://oeis.org/wiki/List_of_LaTeX_mathematical_symbols}{Bedienungsanleitung}

+
-
$\div \ast \times$

\subsubsection{Griechische Symbole}

\href{https://oeis.org/wiki/List_of_LaTeX_mathematical_symbols}{Bedienungsanleitung}

$\alpha$
$\pi$
$\Delta$
$\Theta$
$\vartheta$
$\mu$
$\Lambda$

\subsubsection{Relations Operatoren}

< > =
$\nless$
$\ngtr$
$\parallel$
$\leq$
$\geq$
$\ngeqslant$
$\perp$
$\subset$
$\not\supset$
$\nparallel$
$\subseteq$
$\approx$
$\sim$
$\neq$
$\mid$

\subsubsection{Mathematische Notation}

\href{https://oeis.org/wiki/List_of_LaTeX_mathematical_symbols}{Bedienungsanleitung}

$\mathbb{R}$
$\mathbb{G}$
$\mathbb{C}$
$\in$
$\notin$
$\exists$
$\overline{ab}$
$f'(x)$

\subsubsection{Binär Operatoren}

\href{https://oeis.org/wiki/List_of_LaTeX_mathematical_symbols}{Bedienungsanleitung}

$\neg$ 
$\lor$
$\land$

\newpage

\subsubsection{Bruch}

\href{http://namsu.de/Extra/befehle/Bruch.pdf}{Bedienungsanleitung}

Innerhalb eines Textes $\frac{100}{(x \times y)^2}$ diese Formel.

\[{\frac{100}{(x \times y)^2}}\]

\subsubsection{Wurzeln ziehen}

\href{https://www.namsu.de/Mathematik/1-6-2.html}{Bedienungsanleitung}

Innerhalb eines Textes $\sqrt[n]{\frac{100}{(x \times y)^2}}$ diese Formel.

\[\sqrt[n]{\frac{100}{(x \times y)^2}}\]

\subsubsection{Summenzeichen}

\href{https://de.overleaf.com/learn/latex/Integrals%2C_sums_and_limits}{Bedienungsanleitung}

Innerhalb eines Textes $\sum_{n=0}^{\infty}\frac{10}{n^2+1}$ eine Formel.

\[\sum_{n=0}^{\infty}\frac{10}{n^2+1}\]

\subsubsection{Integral}

\href{https://de.overleaf.com/learn/latex/Integrals%2C_sums_and_limits}{Bedienungsanleitung}

Innerhalb eines Textes $\int_sx^2*\frac{20}{dx}$ eine Formel.

\[\int_sx^2*\frac{20}{dx}\]

\[\iint_{0}^1sx^2\frac{20}{dx}\]

\newpage

\subsubsection{Limes}

\href{https://de.overleaf.com/learn/latex/Integrals%2C_sums_and_limits}{Bedienungsanleitung}

Innerhalb eines Textes $lim_{x\to \infty} x$ eine Formel.

\[lim_{x\to \infty} x\]

\subsubsection{Matrix}

\href{https://www.overleaf.com/learn/latex/Matrices}{Bedienungsanleitung}

Matrix im Text,
  $\big(\begin{smallmatrix}
  a & b\\
  c & d
\end{smallmatrix}\big)$
passt.

$\begin{pmatrix}
  1 & 1 & 1\\
  0 & 0 & 1\\
  1 & 0 & 1
\end{pmatrix}$

\subsubsection{Fälle}

\href{https://www.overleaf.com/latex/examples/cases/nndqpbymnchn}{Bedienungsanleitung}

\[
|x|=
\begin{cases}
x & \text{ für } x < 0\\
x & \text{ für } x > 100\\
0 & \text{ für } x = 0
\end{cases}
\]

\subsubsection{Mathe in mehreren Zeilen}

\begin{align}
  f(x)&=\sqrt[3]{\frac{x}{z^2}} &f(x)&=x^2+bx+q\\
  h(x)&=\frac{n^2}{\sqrt{x-2}} &h(x)&=\sqrt{n}\ast n+f(x)^2\notag
\end{align}

\subsection{Abkürzungen}

\href{https://www.namsu.de/Extra/pakete/Acronym.html}{Bedienungsanleitung (acronym)}

Normal:                                   \ac{DHGE}

Direkte Abkürzung:                        \acs{DHGE}

Direkte Langform:                         \acl{DHGE}

Direkt Langform + Abkürzung:              \acf{DHGE}

Keine Langform + Plural:                  \acp{DHGE}

Direkte Kurzform + Plural:                \acsp{DHGE}

Direkte Langform + Plural:                \acfp{DHGE}

Direkte Langform + Abkürzung + Plural:    \aclp{DHGE}

Abweichender Plural:                      \acp{VM}
\newpage

% ########## Grafiken ########## %

\subsection{Grafiken}

\subsubsection{Bilder}

\href{https://texdoc.org/serve/graphicx.pdf/0}{Bedienungsanleitung (Graphicx)}
\href{https://en.wikibooks.org/wiki/LaTeX/Importing_Graphics}{Bedienungsanleitung (Graphicx)}

\par\medskip
\begin{figure}[!ht]
    \centering
    \frame{\includegraphics[width=\textwidth]{example-image-a}}
    \caption[Titel in Abbildungsverzeichnis]{Titel\footnotemark}
    \label{fig:ExampleImage}
\end{figure}
\footnotetext{QUELLE}

advanced image:

\begin{figure}[!ht]
	\centering
	\begin{tikzonimage}[width=14cm]{example-image-b}[tsx/show help lines] %show help lines used for positioning of elements
	
		\draw [black, thick](0,0) rectangle (1,1); %border
		\draw[red,ultra thick,rounded corners] (0.4,0.4) rectangle (0.999,0.4); %line on picture, other methods definetly possible
		\draw [black, thick](0.28,0.7) rectangle node {1}  (0.32,0.8); %number on picture
		
	\end{tikzonimage}

	\caption{advanced image}
	\label{fig:img}%adressable with \ref{fig:img}
\end{figure}

\newpage

\subsubsection{Bilder neben Text}

\href{https://ctan.ebinger.cc/tex-archive/macros/latex/contrib/wrapfig/wrapfig-doc.pdf}{Bedienungsanleitung (wrapfig)}

\par\medskip
\begin{wrapfigure}[11]{r}[0pt]{5cm}
    \centering
    \frame{\includegraphics[width=0.3\textwidth]{example-grid-100x100pt}}
    \caption[Titel in Abbildungsverzeichnis (wrapfig)]{Titel\footnotemark}
    \label{fig:ExampleFloatImage}
\end{wrapfigure}
\footnotetext{QUELLE}
Lorem ipsum dolor sit amet, consetetur sadipscing elitr, sed diam nonumy eirmod tempor invidunt ut labore et dolore magna aliquyam erat, sed diam voluptua. At vero eos et accusam et justo duo dolores et ea rebum. Stet clita kasd gubergren, no sea takimata sanctus est Lorem ipsum dolor sit amet. Lorem ipsum dolor sit amet, consetetur sadipscing elitr, sed diam nonumy eirmod tempor invidunt ut labore et dolore magna aliquyam erat, sed diam voluptua. At vero eos et accusam et justo duo dolores et ea rebum. Stet clita kasd gubergren, no sea takimata sanctus est Lorem ipsum dolor sit amet. Lorem ipsum dolor sit amet, consetetur sadipscing elitr, sed diam nonumy eirmod tempor invidunt ut labore et dolore magna aliquyam erat, sed diam voluptua. At vero eos et accusam et justo duo dolores et ea rebum. Stet clita kasd gubergren, no sea takimata sanctus est Lorem ipsum dolor sit amet.
Duis autem vel eum iriure dolor in hendrerit in vulputate velit esse molestie consequat, vel illum dolore eu feugiat nulla facilisis at vero eros et accumsan et iusto odio dignissim qui blandit praesent luptatum zzril delenit augue duis dolore te feugait nulla facilisi. Lorem ipsum dolor sit amet, consectetuer adipiscing elit, sed diam nonummy nibh euismod tincidunt ut laoreet dolore magna aliquam erat volutpat.
\newpage

\subsubsection{Flussdiagramm nach DIN 66001}

\href{https://tikz.dev/}{Bedienungsanleitung (tikz)}

\par\medskip
\begin{figure}[!ht]
  \centering
    \begin{tikzpicture}[node distance=2cm]
      \node (start) [startstop] {Start};
      \node (in1) [io, below of=start] {Input};
      \node (pro1) [process, below of=in1] {Process 1};
      \node (dec1) [decision, below of=pro1, yshift=-0.5cm] {Decision 1};
      \node (pro2a) [process, below of=dec1, yshift=-0.5cm] {Process 2a};
      \node (pro2b) [process, right of=dec1, xshift=2cm] {Process 2b};
      \node (out1) [io, below of=pro2a] {Output};
      \node (stop) [startstop, below of=out1] {Stop};
      \draw [arrow] (start) -- (in1);
      \draw [arrow] (in1) -- (pro1);
      \draw [arrow] (pro1) -- (dec1);
      \draw [arrow] (dec1) -- node[anchor=south] {no} (pro2b);
      \draw [arrow] (dec1) -- node[anchor=east] {yes} (pro2a);
      \draw [arrow] (pro2b) |- (pro1);
      \draw [arrow] (pro2a) -- (out1);
      \draw [arrow] (out1) -- (stop);
    \end{tikzpicture}
  \caption[Titel in Abbildungsverzeichnis(Flowchart)]{Titel}
  \label{fig:ExampleFlowchart}
\end{figure}
\newpage

\subsubsection{Code einfügen}

\href{https://texdoc.org/serve/listings.pdf/0}{Bedienungsanleitung (listings)}

\begin{figure}[!ht]
\begin{lstlisting}[language=Python]
import random as randintint 
number = randint(1, 10)
def main(argv):
    return 0
\end{lstlisting}
\caption[Titel in Abbildungsverzeichnis (Code)]{Titel\footnotemark}
\label{fig:ExampleCode}
\end{figure}
\footnotetext{QUELLE}
\newpage

\subsubsection{Tabellen}

\href{https://texdoc.org/serve/tabularx/0}{Bedienungsanleitung (tabularx)}

\begin{table}[!ht]
\begin{tabular}{||c c c c||}
\hline
Spalte 1 & Spalte 2 & Spalte 3 \\ [0.5ex]
\hline\hline
1 & Inhalt & Inhalt \\
2 & Inhalt & Inhalt \\
3 & Inhalt & Inhalt \\
4 & Inhalt & Inhalt \\
5 & Inhalt & Inhalt \\ [1ex]
\hline
\end{tabular}
\caption[Tabellenname im Tabellenverzeichnis]{Tabellentitel\footnotemark}
\label{table:ExampleTable}
\end{table}
\footnotetext{QUELLE}
\newpage

\subsubsection{Ordnerstrukturen}

\href{https://de.mirrors.cicku.me/ctan/macros/generic/dirtree/dirtree.pdf}{Bedienungsanleitung (dirtree)}

\par\medskip
\begin{figure}[!ht]
  \fbox{
  \begin{minipage}{7cm}
  \dirtree{%
.1 Ordner A.
.2 Unterordner B.
.3 Datei D1.
.3 Datei D2.
.2 Unterordner B1.
.3 Datei DD1.
.4 $...$.
.2 Unterordner C.
.3 Datei D3.
.3 Datei D4.
.2 $...$.
}
\end{minipage}
}
\caption[Ordnerstruktur im Abbildungsverzeichnis]{Ordnerstruktur}
\label{fig:JSONStruct}
\end{figure}
\newpage

\subsubsection{Verweisen auf Grafiken}

Siehe Abbildung \ref{fig:ExampleImage}

Siehe Abbildung \ref{fig:ExampleFloatImage}

Siehe Abbildung \ref{fig:ExampleCode} Z. 2ff.

Siehe Abbildung \ref{fig:ExampleFlowchart}

Siehe Tabelle \ref{table:ExampleTable}

Siehe im Anhang \ref{Anhang:1}
\newpage

\subsection{PDF einfügen}

\href{https://texdoc.org/serve/pdfpages.pdf/0}{Bedienungsanleitung (pdfpages)}
%\includepdf[pages={SEITENZAHL}]{DATEINAME.pdf}
\newpage

\subsection{Vorlagen}

\href{https://de.overleaf.com/latex/templates}{Overleaf Vorlagen}

\newpage

% ########## Literaturverzeichnis ########## %

% Römische Seitenzahlen fortsetzen
\pagenumbering{Roman}
\setcounter{page}{\value{savepage}}

\phantomsection
\section*{Literaturverzeichnis}
\addcontentsline{toc}{section}{Literaturverzeichnis}
\renewcommand\refname{}
\printbibliography
\newpage

% ########## Anlagen ########## %
\phantomsection
% Anlagenverzeichnis
\setlength{\cftfigindent}{0em}
\setlength{\cftfignumwidth}{6em}
\renewcommand{\cftfigpresnum}{Anlage }
\listofanlagen
\newpage

% ########## Anlage 1 ########## %

\begin{anlage}[!ht]
  \caption[Titel im Anlagenverzeichnis Anlage 1]{Titel Anlage 1}
  \label{Anhang:1}
  \centering
  \frame{\includegraphics[angle=-90, width=\textwidth]{example-image-a}}
\end{anlage}
\newpage

% ########## Anlage 2 ########## %

\begin{landscape}
  \begin{anlage}[!ht]
    \caption[Titel im Anlagenverzeichnis Anlage 2]{Titel Anlage 2}
    \label{Anhang:2}
    \centering
    \frame{\includegraphics[width=1.2\textwidth]{example-image-a}}
  \end{anlage}
  \end{landscape}
  \newpage

% ########## Anlage 3 ########## %

\begin{anlage}[!ht]
  \caption[Titel im Anlagenverzeichnis Anlage 3]{Titel Anlage 3}
  \label{Anhang:3}
  \centering
  \frame{\includegraphics[angle=-90, width=\textwidth]{example-image-a}}
\end{anlage}
\newpage

% ########## Ehrenwörtliche Erklärung ########## %

\begin{center}
  \textbf{Ehrenwörtliche Erklärung}
\end{center}
\vspace*{1.5cm}
Ich erkläre hiermit ehrenwörtlich,
\begin{flushleft}
  \begin{enumerate}[leftmargin=0.5cm]
    \item 	dass ich meine {Projektarbeit}
    mit dem Thema:  \\
    \vspace*{1cm}
            \textbf{\inserttitle} \\
    \vspace*{1cm}
            ohne fremde Hilfe angefertigt habe, \\
    \item	dass ich die Übernahme wörtlicher Zitate aus der Literatur sowie die Verwendung der
            Gedanken anderer Autoren an den entsprechenden Stellen innerhalb der Arbeit gekennzeichnet habe und  \\
    \item	dass ich meine {Projektarbeit}
    bei keiner anderen Prüfung vorgelegt habe. \\
    \vspace*{1cm}
  \end{enumerate}
  \noindent
Ich bin mir bewusst, dass eine falsche Erklärung rechtliche Folgen haben wird.
\end{flushleft}
\vspace*{1cm}
\begin{tabular} {lrl}
  \hspace{5.5cm} & \hspace{3cm} & \hspace{5.5cm} \\
  \hrulefill & & \hrulefill \\
  Ort, Datum & & Unterschrift
\end{tabular}
\vspace*{\fill}
\end{document}
